\chapter{Summary}
Overall, this report discusses the conversion between the SBML-qual and the ERODE format.  A converter, capable of converting back and forth between these formats, was successfully implemented.  The converter is based upon principles that render the implementation maintainable.  In other words,  one can easily extend and modify the converter if further development is needed.

To achieve this goal, several milestones, representing each stage of the project, were defined. During the first stage, the two formats haven been analyzed, to become familiar with them and obtain an overview of the conversion requirements. In the second and third stage the knowledge about the conversion requirements was then used to design and implement the converter application. Initially, a few small demo-applications were created to develop a framework for the final application. This framework was then extended into the final converter by adding new functionality iteratively. During each iteration, each new piece of functionality was tested thoroughly to ensure that the formats are converted correctly. An important part of the converter design was also to structure it in a way, that allows an integration into other host programs.

The development process was guided by a project plan to track the overall progress of the project. Additionally, weekly meetings with the project supervisors were used to discuss various aspects of the project in detail, which helped to tackle problems early and ensured progression in the project.