% Colours! 
\newcommand{\targetcolourmodel}{cmyk} % rgb for a digital version, cmyk for a printed version. Only use lowercase
\selectcolormodel{\targetcolourmodel}

% Define colours from https://www.designguide.dtu.dk/
\definecolor{dtured}    {rgb/cmyk}{0.6,0,0 / 0,0.91,0.72,0.23}
\definecolor{blue}      {rgb/cmyk}{0.1843,0.2431,0.9176 / 0.88,0.76,0,0}
\definecolor{brightgreen}{rgb/cmyk}{0.1216,0.8157,0.5098 / 0.69,0,0.66,0}
\definecolor{navyblue}  {rgb/cmyk}{0.0118,0.0588,0.3098 / 1,0.9,0,0.6}
\definecolor{yellow}    {rgb/cmyk}{0.9647,0.8157,0.3019 / 0.05,0.17,0.82,0}
\definecolor{orange}    {rgb/cmyk}{0.9882,0.4627,0.2039 / 0,0.65,0.86,0}
\definecolor{pink}      {rgb/cmyk}{0.9686,0.7333,0.6941 / 0,0.35,0.26,0}
\definecolor{grey}      {rgb/cmyk}{0.8549,0.8549,0.8549 / 0,0,0,0.2}
\definecolor{red}       {rgb/cmyk}{0.9098,0.2471,0.2824 / 0,0.86,0.65,0}
\definecolor{green}     {rgb/cmyk}{0,0.5333,0.2078 / 0.89,0.05,1,0.17}
\definecolor{purple}    {rgb/cmyk}{0.4745,0.1373,0.5569 / 0.67,0.96,0,0}

\newcommand{\dtulogocolour}{white} % Colour of the DTU logo: white, black or dtured
\newcommand{\frontpagetextcolour}{white} % front page text colour: white or black
\colorlet{frontbackcolor}{blue} % Set the background colour of the front- and back page. Choose the colour so it matches the main colour of front page picture

% DTU colours for diagrams
% You might want to make the front/back page background colour the first colour in the plot cycle list.
\pgfplotscreateplotcyclelist{DTU}{%
dtured,         fill=dtured,        \\%
blue,           fill=blue,          \\%
brightgreen,    fill=brightgreen    \\%
navyblue,       fill=navyblue       \\%
yellow,         fill=yellow         \\%
orange,         fill=orange         \\%
grey,           fill=grey           \\%
red,            fill=red            \\%
green,          fill=green          \\%
purple,         fill=purple         \\%
}


% Font
% There is no corporate serif font in the DTU design guide. The DTU design team has proposed to use Neo Sans for headings - and Arial for the body text.
% To change heading font to NeoSans Pro please upload both NeoSansPro-Regular.otf and NeoSansPro-Medium.otf to the root directory.
\setmainfont{Arial}
\renewcommand\thepart{Part \Roman{part}}
\IfFontExistsTF{NeoSansPro-Medium.otf}
{ %If True set headings to NeoSans Pro
\newfontface\NeoSansProReg{NeoSansPro-Regular.otf}
\newfontface\NeoSansProMed{NeoSansPro-Medium.otf}
\titleformat{\part}[display]{\NeoSansProMed \huge \centering}{\NeoSansProMed \Huge \thepart}{1em}{\thispagestyle{empty}}{}
\titleformat{\chapter}{\NeoSansProMed\huge}{\thechapter}{1em}{\raggedright}
\titleformat{\section}{\NeoSansProMed\Large}{\thesection}{1em}{\raggedright}
\titleformat{\subsection}{\NeoSansProMed\large}{\thesubsection}{1em}{\raggedright}
\titleformat{\subsubsection}{\NeoSansProMed\normalsize}{\thesubsubsection}{1em}{\raggedright}
\newcommand\TitleFont[1]{{\NeoSansProMed #1}}
\newcommand\titlefont[1]{{\NeoSansProReg #1}}
}
{ % If false
\titleformat{\part}[display]{\bfseries\huge \centering}{\bfseries\Huge \thepart}{1em}{\thispagestyle{empty}}{}
\titleformat{\chapter}{\bfseries\huge}{\thechapter}{1em}{\raggedright}
\titleformat{\section}{\bfseries\Large}{\thesection}{1em}{\raggedright}
\titleformat{\subsection}{\bfseries\large}{\thesubsection}{1em}{\raggedright}
\titleformat{\subsubsection}{\bfseries\normalsize}{\thesubsubsection}{1em}{\raggedright}
\newcommand\TitleFont[1]{{\bfseries #1}}
\newcommand\titlefont[1]{{#1}}
}
\urlstyle{sf}
%\def\UrlFont{\NeoSansProReg}


% If you wish to use the pdflatex compiler, sans-serif Helvetica can be used as a replacement for Arial. In this way you will be able to use the microtype package. Remember to add \usepackage[utf8]{inputenc} and disable the fontspec package. 
%\newcommand\TitleFont[1]{{\bfseries #1}}
%\newcommand\titlefont[1]{{#1}}
%\fontfamily{qhv}\selectfont
%\renewcommand{\familydefault}{\sfdefault}


% Watermark for confidential or draft (or anything else)
%\sffamily % set the correct font for the watermark
\newsavebox\mybox
\savebox\mybox{\tikz[color=grey,opacity=0.5]\node{Template};}

\newwatermark*[
  oddpages,
  angle=60,
  scale=12,
  %fontfamily=qhv,
  xpos=-40,
  ypos=30,
]{\usebox\mybox}

\newwatermark*[
  evenpages,
  angle=60,
  scale=12,
  %fontfamily=qhv,
  xpos=-50,
  ypos=30,
]{\usebox\mybox}


% Table of contents (TOC) and numbering of headings
\setcounter{tocdepth}{1}    % Depth of table of content: sub sections will not be included in table of contents
\setcounter{secnumdepth}{2} % Depth of section numbering: sub sub sections are not numbered

\makeatletter % Reset chapter numbering for each part
\@addtoreset{chapter}{part}
\makeatother  

% Spacing of titles and captions
\titlespacing\chapter{0pt}{0pt plus 4pt minus 2pt}{4pt plus 2pt minus 2pt}
\titlespacing\section{0pt}{12pt plus 3pt minus 3pt}{2pt plus 1pt minus 1pt}
\titlespacing\subsection{0pt}{8pt plus 2pt minus 2pt}{0pt plus 1pt minus 1pt}
\titlespacing\subsubsection{0pt}{4pt plus 1pt minus 1pt}{-2pt plus 1pt minus 1pt}
\captionsetup{belowskip=\parskip,aboveskip=4pt plus 1pt minus 1pt}

% Setup header and footer
\fancypagestyle{main}{% All normal pages
    \fancyhead{}
    \fancyfoot{}
    \renewcommand{\headrulewidth}{0pt}
    \fancyfoot[LE,RO]{\footnotesize \thepage}
    \fancyfoot[RE,LO]{\footnotesize \thesistitle} % - \rightmark
    \fancyhfoffset[E,O]{0pt}
}
\fancypagestyle{plain}{% Chapter pages
    \fancyhead{}
    \fancyfoot{}
    \renewcommand{\headrulewidth}{0pt}
    \fancyfoot[LE,RO]{\footnotesize \thepage}
    \fancyfoot[RE,LO]{\footnotesize \thesistitle} % - \leftmark
    \fancyhfoffset[E,O]{0pt}
}


% Setup for diagrams and graphs (tikz pictures) 
\usetikzlibrary{spy}    % For magnifying anything within a tikzpicture, see the line graph
\usepgfplotslibrary{statistics} % Package for the boxplot
\pgfplotsset{ % Setup for diagrams
compat=newest,
major x grid style={line width=0.5pt,draw=grey},
major y grid style={line width=0.5pt,draw=grey},
legend style={at={(0.5,-0.1)}, anchor=north,fill=none,draw=none,legend columns=-1,/tikz/every even column/.append style={column sep=10pt}},
axis line style={draw=none},
tick style={draw=none},
every axis/.append style={ultra thick},
tick label style={/pgf/number format/assume math mode}, % To apply main font to tick labels (numbers on the axis)
}
\tikzset{every mark/.append style={scale=1.5}}


% Hypersetup
\hypersetup{
    pdfauthor={\thesisauthor},
    pdftitle={\thesistitle},
    pdfsubject={\thesissubtitle},
    pdfdisplaydoctitle,
    bookmarksnumbered=true,
    bookmarksopen,
    breaklinks,
    linktoc=all,
    plainpages=false,
    unicode=true,
    colorlinks=false,
    hidelinks,                        % Do not show boxes or coloured links.
}


% Listings setup
\lstset{
    basicstyle=\footnotesize\ttfamily,% the size of the fonts that are used for the code
    commentstyle=\color{green},       % comment style
    keywordstyle=\bfseries\ttfamily\color{blue}, % keyword style
    numberstyle=\sffamily\tiny\color{grey}, % the style that is used for the line-numbers
    stringstyle=\color{purple},       % string literal style
    rulecolor=\color{grey},           % if not set, the frame-color may be changed on line-breaks within not-black text (e.g. comments (green here))
    breakatwhitespace=false,          % sets if automatic breaks should only happen at whitespace
    breaklines=true,                  % sets automatic line breaking
    captionpos=b,                     % sets the caption-position to bottom
    deletekeywords={},                % if you want to delete keywords from the given language
    escapeinside={\%*}{*)},           % if you want to add LaTeX within your code
    frame=single,                     % adds a frame around the code
    xleftmargin=4pt, 
    morekeywords={*,...},             % if you want to add more keywords to the set
    numbers=left,                     % where to put the line-numbers; possible values are (none, left, right)
    numbersep=10pt,                   % how far the line-numbers are from the code
    showspaces=false,                 % show spaces everywhere adding particular underscores; it overrides 'showstringspaces'
    showstringspaces=false,           % underline spaces within strings only
    showtabs=false,                   % show tabs within strings adding particular underscores
    stepnumber=1,                     % the step between two line-numbers. If it's 1, each line will be numbered
    tabsize=2,                        % sets default tabsize to 2 spaces
    title=\lstname,                   % show the filename of files included with \lstinputlisting; also try caption instead of title
}

% Signature field
\newlength{\myl}
\newcommand{\namesigdatehrule}[1]{\par\tikz \draw [black, densely dotted, very thick] (0.04,0) -- (#1,0);\par}
\newcommand{\namesigdate}[2][]{%
\settowidth{\myl}{#2}
\setlength{\myl}{\myl+10pt}
\begin{minipage}{\myl}%
\begin{center}
    #2  % Insert name from the command eg. \namesigdate{\authorname}
    \vspace{1.5cm} % Spacing between name and signature line 
    \namesigdatehrule{\myl}\smallskip % Signature line and a small skip
    \small \textit{Signature} % Text under the signature line "Signature"
    \vspace{1.0cm} % Spacing between "Signature" and the date line
    \namesigdatehrule{\myl}\smallskip % Date line and a small skip
    \small \textit{Date} % Text under date line "Date" 
\end{center}
\end{minipage}
}

% For the back page: cleartoleftpage
\newcommand*\cleartoleftpage{%
  \clearpage
  \ifodd\value{page}\hbox{}\newpage\fi
}
